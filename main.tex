\documentclass{mipt-thesis-bs}
    % Следующие две строки нужны только для biblatex. Для inline-библиографии их следует убрать.
    %\usepackage{mipt-thesis-biblatex}
    %\addbibresource{bib.bib}

\title{Спектры предложений первого порядка с ограниченным количеством переменных}
\author{Ярмошик Д.\,В.}
\supervisor{Жуковский М.\,Е.}
%\referee{Петров Д.\,Е.}       % требуется только для mipt-thesis-ms
\groupnum{573}
\faculty{Факультет управления и прикладной математики}
\department{Кафедра «Интеллектуальные системы», специализация «Проектирование и организация систем»}

\begin{document}

\frontmatter
\titlecontents

\mainmatter

\chapter{Аннотация}
В данной работе исследуются асимптотические свойства случайных графов модели Эрдеша-Реньи и свойства языков первого порядка с ограниченным числом переменных на графах. Рассматриваются вопросы о спектрах предложений из этих языков и о выполнении законов нуля или единицы для этих языков. В работе доказано существование формулы с бесконечным спектром из $\LL^3_{\infty, \omega}$, доказано, что объединение спектров всех формул из $\LL^3$ имеет бесконечное число предельных точек в левой окрестности единицы.

\textit{Ключевые слова:} случайные графы, модель Эрдёша-Реньи, языки первого порядка

\chapter{Введение}
    Случайные графы -- один из центральных разделов дискретной математики, расположенный на стыке теории вероятностей, комбинаторики и теории графов.
    Основы теории случайных графов были заложены в 50-х -- 60-х годах прошлого века венгерскими математиками П.~Эрдёшем и А.~Реньи.
    Существует множество моделей случайных графов, разработанных с учётом адекватности их применения в прикладных областях: моделирования социальных, биологических и инфраструктурных сетей. 
    В настоящей работе мы будем иметь дело с \textit{моделью Эрдёша--Реньи}, также называемой \textit{классической моделью} случайного графа.
    % Определение G(N,p)
    
    Многие вопросы этой теории связаны с асимптотическими свойствами случайного графа, то есть с тем, как он ведёт себя при устремлении количества вершин $N$ к бесконечности.
    Удивительным образом оказывается, что для целого класса таких вопросов можно, совершенно не вдаваясь в сущность конкретного вопроса, дать на него едва ли не исчерпывающий ответ.
    Точнее говоря, если про случайный граф спрашивается ``Как при увеличении числа вершин до бесконечности будет вести себя вероятность того, что граф обладает свойством $A$?'', то во многих случаях можно заранее сказать: ``Будет стремиться к 0 или к 1''.
    В этом состоит суть теорем, называемых \textit{законами нуля или единицы}, уточнению условий применимости которых посвящена данная работа.
    
    Чтобы перейти к изложению известных результатов о законах нуля или единицы, дадим определения основным понятиям и сформулируем теоремы, касающиеся важных для нас свойств случайного графа $G(N,p)$ (см. \cite{survey2015}).
    
    Пусть $N \in \N$, $0 \leq p \leq 1$. Рассмотрим множество $\Omega_N = \{G = (V_N , E)\}$, $|\Omega_N| = 2^{C_N^2}$ всех неориентированных графов без петель и кратных ребер с множеством вершин $V_N = \{1, \ldots, N \}$.
    
    %NOTE: Максим: Случайные графы бывают независимы, чего нельзя сказать про элементарные исходы
    % \Def \textit{Случайный граф $G(N, p) \in \Omega_N$  в модели Эрдёша-Реньи} -- это элементарный исход в вероятностном пространстве $(\Omega_N, \P_{N.p}, 2^{\Omega_N})$, где  распределение $\P_{N.p}$ определяется формулой $\P_{N.p}(G) = p^{|E|}(1-p)^{C_N^2-|E|}$
    
  \Def \textit{Случайный граф $G(N, p)$  в модели Эрдёша-Реньи}~--- это случайный элемент со значениями во множестве $\Omega_N$ и распределением $\P_{N.p}(G) = p^{|E|}(1-p)^{C_N^2-|E|}$, определённом на $\mathscr{F}=2^{\Omega_N}$.
  
  Иначе говоря, это полный граф на $N$ вершинах, в котором каждое ребро независимо от других рёбер удаляется с вероятностью $1-p$.
  Параметр $p$ называют вероятностью проведения ребра.
    
    \Def Для произвольного языка $\F$ , случайный граф $G(N, p)$ \textit{подчиняется закону нуля или единицы для языка $\F$},
    если для любой формулы $\varphi$ из языка 
    $\F$  выполнено
    $\lim_{N \rightarrow \infty} \P(G(N, p) \vDash \varphi) \in \{0, 1\}$.
    
    Введём обозначение $\LL$ для языка первого порядка с сигнатурой, в которую входят только двуместные предикаты равенства ($=$) и смежности ($\sim$).
    За точными определениями можно обратиться, например, к
    \cite{shen}, а здесь мы ограничимся лишь напоминанием о том, что формулы языка первого порядка~--- это предложения, составленные из символов, обозначающих переменные: 
    $x,y,z,x_1,\ldots$, 
    логических связок 
    $\wedge, ~\neg, ~\vee$, 
    кванторов 
    $\exists, ~\forall$ и 
    предикатных символов (в нашем случае
    $\sim, ~ =$).
    Например, формула, утверждающая, что диаметр графа (расстояние между двумя наиболее удалёнными друг от друга вершинами, измеряемое в рёбрах) не превосходит 2, выглядит так:
    \[
    \forall x \forall y ~ x=y \vee x\sim y \vee \left( \exists z ~ x \sim z \wedge z \sim y \right)
    \]
    
    Всюду далее $E(G)$ будет обозначать множество вершин графа $G$, а $V(G)$~--- множество его рёбер.
    
    \Def \textit{Плотность} графа $G$ 
   \[ \rho(G) = \frac {|E(G)|}{|V(G)|} .\]
   
   \Def \textit{Максимальная плотность} графа $G$
     \[ \rho^{max}(G) = \max_{H \subseteq G} \frac {|E(H)|}{|V(H)|} .\]
    
    \Def Граф $G$ называется \textit{сбалансированным}, если 
    \[ \rho^{max}(G) = \rho(G) .\]
    
    \Def Граф $G$ называется \textit{строго сбалансированным}, если 
    \[\rho^{max}(G) = \rho(G) > \max_{H \subsetneq G} \frac {|E(H)|}{|V(H)|} .\]
     
В 1960 г. П. Эрдёш и А. Реньи
доказали теорему о об условиях справедливости утверждения ``$G(N, p)$ содержит копию данного сбалансированного графа'' \cite{erdHos1976evolution}. 
Позже этот результат получил следующее обобщение.
    
\begin{theorem} (А. Ручински, А. Винс, 1985, \cite{rucinski1985balanced}) 
\label{th:ruchinski}
Вероятность того, что $G(N, p)$ имеет подграф $H$ стремится к 1 при $N \rightarrow \infty$, если $N^{-1/\rho^{max}(H)} = o(p(N))$ и стремится к  0, если $p(N) = o\left(N^{-1/\rho^{max}(H)}\right)$.

\end{theorem}

Для нас очень важно то, что при $p = N^{-1/\rho^{max}(H)}$ вероятность содержать подграф $H$ не стремится ни к нулю, ни к единице.
Это утверждение, содержащееся в следующей теореме, даёт удобный способ доказывать отсутствие законов нуля или единицы для $G(N, N^{-\alpha}), ~\alpha \in \Q$.

\begin{theorem} (Б. Боллобаш, 1981, \cite{bollobas1981threshold}). Пусть $H$~--- строго сбалансированный граф, $a$~--- количество автоморфизмов графа $H$, $p = N^{-1/ \rho^{max}(H)}$
Тогда
\[N_H \xrightarrow[N\rightarrow \infty]{d} Poiss(1/a) \]
Здесь $N_H$~--- количество копий графа $H$ в $G(N, p)$, $Poiss(1/a)$~--- пуассоновская случайная величина со средним $1/a$.
\end{theorem}
Из последней теоремы и доказанного в \cite{rucinski1986strongly} утверждения о том, что для любого $\alpha \in (0,1] \cap \Q$ существует строго сбалансированный граф $H$ с плотностью $\rho(H) = 1/\alpha$, следует, что $G(N,N^{-\alpha})$ не подчиняется закону нуля или единицы для языка $\LL$ при всех рациональных $\alpha$ из интервала $(0,1]$.
Это верно и для $\alpha = (k+1)/k,~ k > 0$, т.к. для каждого из таких $\alpha$ существует дерево с соответствующей плотностью.
При $\alpha > 2$ граф $G(N, N^{-\alpha})$ асимптотически почти наверное пуст, и, следовательно, подчиняется закону нуля или единицы.
Оказывается, что при всех прочих $\alpha$ случайный граф $G(N, N^{-\alpha})$ также подчиняется закону нуля или единицы.
Итак, имеет место следующая теорема
\begin{theorem} (Дж. Спенсер, С. Шелах, 1988, \cite{shelah1988zero})
Случайный граф $G(N, N^{-\alpha})$ подчиняется закону нуля или единицы для языка $\LL$ при всех $\alpha$, кроме $\alpha \in \left((0,1] \cap \Q \right) \cup \{(k+1)/k ~|~ k \in \N \}$.
\end{theorem}

Ниже мы сформулируем несколько аналогичных теорем для других языков.

\Def Язык $\LL^k$~--- подмножество $\LL$, содержащее предложения, в которые входят не более $k$ переменных.

На языке $\LL^3$ для любого натурального $d$ можно выразить, например, свойство ``иметь диаметр не более $d$'': 
\[
\psi_d = \forall x \forall y ~ x = y \vee x \sim y \vee \left( \exists z ~ x\sim z \wedge z \sim y \right)
\vee  \left( \exists z ~ x\sim z \wedge \left( \exists x ~ z \sim x \wedge x \sim y \right) \right) \vee \ldots
\]

\Def Язык $\LL^k_{\infty, \omega}$ включает в себя предложения конечной или счётной длины, в которые входят не более $k$ переменных.

На языке $\LL^3_{\infty, \omega}$ можно выразить, например, свойство ``быть связным'': $\psi = \bigvee_{d=1}^\infty \psi_d$.
Это свойство невыразимо в $\LL$.

\Def \[\LL^\omega_{\infty, \omega} = \bigcup_k \LL^k_{\infty, \omega} \]

\begin{theorem}
Случайный граф $G(N, N^{-\alpha})$ подчиняется закону нуля или единицы для языка $\LL^\omega_{\infty, \omega}$ при $\alpha \in (1,2] \setminus \{(k+1)/k ~|~ k \in \N \}$ (Дж. Линч, 1993  \cite{lynch1993infinitary}) и $\alpha > 2$.
Закон нуля или единицы не выполнен при $\alpha \in (0,1]$ (С. Шелах, 2017 \cite{shelah2017failure})  и $\alpha \in \{(k+1)/k ~|~ k \in \N \}$.
\end{theorem}

\begin{theorem} (М. MакАртур, 1997 \cite{mcarthur1997asymptotic})
Для любого $\alpha < \dfrac{1}{k-1}$ случайный граф $G(N, N^{-\alpha})$ подчиняется закону нуля или единицы для языка $\LL^k_{\infty, \omega}$.
\end{theorem}

Ниже дано неформальное определение кванторной глубины формулы.
Формальное определение есть, например, в \cite{shen}.

\Def \textit{Кванторная глубина} формулы~--- наибольшее число вложенных кванторов в формуле.

\Def Язык $\LL_k$~--- подмножество $\LL$, включающее формулы с кванторной глубиной не более $k$.

\begin{theorem}(М. Е. Жуковский, 2012, \cite{zhukovskii2012zero}). Пусть $p=N^{-\alpha}$, $0 < \alpha < 1/(k - 2)$. 
Тогда случайный граф $G(N, p)$ подчиняется закону нуля или единицы для языка $\LL_k$.
\end{theorem}

\chapter{Постановка задачи}
Нас интересуют выразительные способности языков $\LL^k$ и $\LL^k_{\infty, \omega}$ и, в частности, наиболее простых среди них -- языков  $\LL^3$ и $\LL^3_{\infty, \omega}$.
Из результата МакАртура \cite{mcarthur1997asymptotic} следует, что для этих языков выполнен закон нуля или единицы при $\alpha < \frac{1}{2}$.
Кроме того, известно (М. Жуковский, А. Раджафимахатратра, 2019 \cite{razafimahatratra2019zero}), что при $\alpha = \frac{1}{k-1}$ закон нуля или единицы для языка $\LL^k$ всё ещё справедлив, но в любой правой полуокрестности $\frac{1}{k-1}$ существует $\alpha$, для которого закон нуля или единицы нарушается.
Что можно сказать о сохранении закона нуля или единицы для языков с тремя переменными при других значениях параметра $\alpha$, например в окрестности $\alpha=1$ -- точки, при переходе через которую граф теряет связность?
Существуют утверждения, не выразимые на языке $\LL$, но выразимые в $\LL^3_{\infty, \omega}$ и наоборот.
Для языка $\LL$ закон нуля или единицы выполнен при всех иррациональных $\alpha$, верно ли это для $\LL^3_{\infty, \omega}$?
Различается ли структура множеств параметров, на которых нарушается закон нуля или единицы, у языков с ограниченным количеством переменных и языков с ограниченной кванторной глубиной? 

\chapter{Полученные результаты}
\section{Формула из $\LL^3_{\infty, \omega}$ с бесконечным спектром}
\input{parts/3_infinite_spectrum.tex}

\section{Предельные точки $\LL^3$ в левой окрестности единицы}
\Def Назовём \textit{предельной точкой языка $\F$} такое $\alpha$, что в любой окрестности $\alpha$ существует $\alpha'$, для которого не выполнен закон нуля или единицы для языка $\F$.

Докажем, что в языке $\mathcal{L}^3$ предельные точки существуют в любой левой полуокрестности единицы: 
$\forall \varepsilon > 0 ~ \exists \alpha \in (1 - \varepsilon, 1)$.

Рассмотрим граф, состоящий из $k+2$ треугольников, соединённых простыми путями.
Причём, расстояние (в рёбрах) между вторым и третьим треугольниками, третьим и четвёртым и т. д., одинаково.
Обозначим это расстояние $m$, а расстояние между первым и вторым треугольником --- $s$.

\begin{figure}[h]
    \centering
  \includegraphics[scale=0.4]{picrel/index.png}
  \caption{Граф с $k = 2$, $s = 2$ и $m = 4$}
  \label{fig:chain1}
\end{figure}

Плотность такого графа равна 
$$\rho({H_{ksm}}) = \dfrac{6+s + (3+m)k}{5+s + (2+m)k}$$
и при увеличении $k$ стремится к $\rho_\infty = \dfrac{3+m}{2+m}$.

Граф с $k=0$ состоит из двух треугольников и имеет вид гантели.

Т.к. 
$sign\left(\dfrac{\partial\rho({H_{ksm}})}{\partial k}\right) = sign(3+s-m)$,
то при $m < 3 + s$ плотность графа растёт при увеличении  $k$ и, как нетрудно убедиться, при этом выполнено равенство
$\rho({H_{ksm}}) = \rho^{max}({H_{ksm}})$.
Будем считать $m = s+2$, т.к. при таком соотношении длины гантели и длины добавочных звеньев
граф получается оптимальным с точки зрения плотности, что пригодится в дальнейшем.

Таким образом, для доказательства утверждения достаточно предъявить для каждого такого графа формулу из  $\mathcal{L}^3$, выражающую его существование.

% Для удобства дальнейших рассуждениях введём следующее понятие.
% Пусть имеется граф $G$.

% \Def \textit{Подстановкой} называется отображение, которое ставит в соответствие каждой переменной формулы в зоне действия квантора по этой переменной некоторую вершину графа $G$.

% При фиксированной подстановке одной переменной в зонах действия разных кванторов по этой переменной  могут соответствовать разные вершины.

Вместо 
$\exists x \exists y \exists z ~ x \sim y \wedge y \sim z \wedge x \sim z$
будем писать 
$\triangle_{xyz}$

Рассмотрим формулу
\[ \varphi_{0s} = \triangle_{xyz} \wedge \psi_s^{x \xi},\]
\begin{equation}
\label{f:gantel}
\begin{split}
\psi_s^x &= \exists x ~ 
    \left[ x \sim z \wedge x \neq y \wedge x \nsim y  \right] \\
&\wedge \exists y ~
    \left[ y \sim x \wedge y \neq z \wedge y \nsim z \right] \\
&\wedge \exists z ~
    \left[ z \sim y \wedge z \neq x \wedge z \nsim x  \right] \\
&\wedge ~ \qquad \qquad \cdots \\
&\wedge \exists \xi ~ 
    \xi \sim \nu \wedge \xi \sim \eta,
\end{split}
\end{equation}
где количество выражений в квадратных скобках в $\psi_{s}^{x}$ равно $s+1$; верхний индекс $x$ соответствует переменной, стоящей после первого квантора существования, вместо символа $\xi$ стоит одна из переменных $x,y,z$ --- какая именно определяется параметром $s$ (переменные у кванторов существования следуют в циклическом порядке: $x,y,z,x,y, \ldots$); $\nu$ и $\eta$ -- две оставшиеся из $x,y,z$ переменные, отличные от $\xi$. 

Если бы в этой формуле после каждого нового квантора существования присутствовало условие на то, что переменная, стоящая после квантора не равна всем переменным, объявленным ранее, то $\varphi_{0s}$ выражала бы существование гантели c расстоянием между треугольниками, равным $s$.
К сожалению, имея лишь три переменных, мы не можем напрямую запретить одной переменной соответствовать нескольким вершинам, поэтому $G \vDash \varphi_{0s}$ не только в случае, когда граф $G$ содержит гантель соответствующей длины (например, граф, изображённый на на Рис. \ref{fig:reuse intermediate}, удовлетворяет $\varphi_{0,6}$).
Впрочем, как мы покажем далее, делать это и не обязательно.

Составим теперь аналогичную формулу $\varphi_{ksm}$ для графа $H_{ksm}$.

\[ \varphi_{ksm} =  \triangle_{xyz} \wedge \psi_{s}^{\xi_0} \wedge \underbrace{\psi_{m}^{\xi_1} \wedge \ldots \wedge \psi_{m}^{\xi_{k}}}_{\text{k раз}} \]

% \begin{equation}
% \label{f:phiKsm}
% \left. 
%     \begin{split}
%         \exists x~\triangle_{xyz} 
%         \gantelRightPart
% \right\}
% \left. 
%     \left. \gantelRightPart \right\}
%     \left. \gantelRightPart \right\}
% \right\}
% \end{equation}

% \[
% \begin{split}
% \left.
% a = b \\
% c = d
% \right\}
% \end{split}
% \]

По той же причине, существование в $G(n, n^{-\alpha})$ подграфа $H_{ksm}$ не является необходимым для того, чтобы формула $\varphi_{ksm}$ была верна.
Однако, оно становится необходимым (c вероятностью, стремящейся к единице) при $\alpha = 1/\rho_{H_{ksm}}$ и $m = s+2$. 
Действительно, покажем, что любой другой граф $\tilde H_{ksm}$, для которого выполнено $\varphi_{ksm}$, содержит подграф ограниченного размера с максимальной плотностью большей, чем $\rho_{H_{ksm}}$.

\begin{Lem} 
\label{lem:min_ro_Hksm}
% Граф $H_{ksm}$ имеет минимальную $\rho^{max}$ среди всех графов, для которых выполнено $\varphi_{ksm}$.
Если $G \vDash \varphi_{ksm}$ и $G$ не содержит подграфа, изоморфного $H_{ksm}$, то $G$ содержит более плотный подграф с меньшим числом вершин.
\end{Lem}

\begin{proof}
В графе $H_{ksm}$ каждой переменной формулы $\varphi_{ksm}$ соответствует отдельная вершина.
Пусть для графа $G$ истинна $\varphi_{ksm}$, и он не содержит подграфа, изоморфного $H_{ksm}$.
% то он имеет вершину, которой соответствует более одной переменной.
% Далее будем рассматривать только второй случай.
% Пусть в звене под номером $k'$ (номер $0$ соответствует гантели), впервые произошло повторное ``использование'' вершины. 

$G \vDash \varphi_{ksm}$
означает, что существует такое отображение из множества вершин графа на множество переменных* формулы, что если в формулу вместо переменных подставить соответствующие вершины и убрать кванторы, то получится верное утверждение.
Здесь мы под \textit{переменной*} понимаем переменную в зоне действия квантора по этой переменной.
Так, одной букве-переменной $x$ соответствует несколько переменных со звёздочкой.
Зафиксируем одно из таких отображений.
Обозначим $\tilde H_{ksm}$ подграф графа $G$, индуцированный на прообразе этого отображения.
Очевидно, что некоторой вершине $\tilde H_{ksm}$ соответствует более одной переменной, т.к.  иначе $\tilde H_{ksm}$ изоморфен $H_{ksm}$.
Назовём конъюнкт вида $\psi^{\xi_l}_*$, $* \in \{s, m \}$, $l$-ой \textit{секцией} формулы $\varphi_{ksm}$.
% Почётче и с примером
Пусть $k'$ -- номер секции, в которой впервые произошло повторное ``использование'' вершины, то есть в $(\triangle_{xyz} \wedge \psi_{s}^{\xi_0} \wedge \psi_{m}^{\xi_1} \wedge \ldots \wedge \psi_{m}^{\xi_{k'}})$ есть две переменные, соответствующие одной вершине, а в $(\triangle_{xyz} \wedge \psi_{s}^{\xi_0} \wedge \psi_{m}^{\xi_1} \wedge \ldots \wedge \psi_{m}^{\xi_{k'-1}})$ таких двух переменных нет.
% для каждого $k' \leq k$ определим $\tilde H'_{k'sm}$ как подграф $\tilde H_{ksm}$, индуцированный на множество вершин, соответствующее переменным, входящим в подформулу формулы $\varphi_{ksm}$, совпадающую с $\varphi_{k'sm}$.
% Пусть теперь $k' \in $ такое, что в $\tilde H_{k'sm}$ есть вершина, которой соответствуют две переменные, а (если $k' \neq 0$) в $~\tilde H_{k'-1,sm}$ таких вершин нет.\\
$e:= |E(H_{k'sm})|$ , $v:= |V(H_{k'sm})|$\\
$\tilde e:= |E(\tilde H_{k'sm})|$ , $\tilde v:= |V(\tilde H_{k'sm})|$\\
Есть три возможности:
\begin{enumerate}
\item 
Последняя вершина в звене уже была использована --- та, которая сразу добавляет два ребра
\begin{enumerate}
\item 
повторно использована только она: $\tilde \rho^{max} \geq \dfrac{e}{v-1},$ 
так как в этом случае число вершин в $\tilde H_{k'sm}$ равно $v-1$, а число рёбер не меньше $e$. Действительно, в рассматриваемом случае (по определению) $\tilde H_{k'sm}$ содержит все рёбра $H_{k'sm}$, кроме, быть может, двух рёбер, инцидентных последней вершине (вершине, соответствующей последней переменной секции с номером $k'$), которые могут уже присутствовать в $H_{k'sm}$.
Однако, легко видеть, что это невозможно (см. Рис. \ref{fig:only last reused}).
\begin{figure}
  \centering
  \includegraphics[scale=0.5]{picrel/only_last_reused.png}
  \caption{Только последняя вершина использована повторно. Черным обозначены вершины и рёбра $\tilde H_{k'sm}$, серым вершины и рёбра  $H_{k'sm}$, красным пунктиром -- рёбра, запрещённые формулой  }
  \label{fig:only last reused}
\end{figure}
\item
она и какая-то до неё: $\tilde \rho^{max} \geq \dfrac{e-2}{v-2}.$ 
\begin{figure}
  \centering
  \includegraphics[scale=0.5]{picrel/reuse_intermediate.png}
  \caption{Повторное использование промежуточной вершины}
  \label{fig:reuse intermediate}
\end{figure}
\label{enum:reuse intermediate}
Действительно, посмотрим на первую повторно использованную вершину.
Т.к. формула запрещает совпадать вершинам, соответствующим разным переменным, для её соединения с предыдущими вершинами необходимо добавить ребро, которого нет в графе $H_{k'sm}$. 
С каждой новой секцией в $H_{k'sm}$ добавляется $3+m$ рёбер  и $2+m$ вершин.
Однако, проведя дополнительное ребро к уже имеющейся вершине, мы уже добавили на одно ребро больше, чем вершин, поэтому имеем рёбер $ \tilde e' = e - (3+m) + (\nu+1) = e - (m-\nu) $, вершин $\tilde v' = v - (2+m) + \nu = v - (m-\nu)$, где $\nu$ -- число вершин, уже добавленных в последней ``секции'' $\tilde H_{k'sm}$.
Поскольку $\tilde H_{k'sm}$ связен, при добавлении новой вершины добавляется и как минимум одно ребро, поэтому $\tilde v = \tilde v' + \mu$, $\tilde e \geq \tilde e' + \mu $ , $\mu > 0$ -- число вершин добавленных после повторного использования.
В рассматриваемом случае $\tilde v \leq v-2$, поэтому $\tilde \rho^{max} \geq \frac{e-2}{v-2}$ 
\end{enumerate}
\item
Последняя вершина ещё не была использована:
$\tilde \rho^{max} \geq \dfrac{e}{v-1}.$
Этот случай похож на \ref{enum:reuse intermediate}, но теперь $\tilde v \leq v-1$ и последняя вершина добавляет 2 ребра
\end{enumerate}

Итого 
$\tilde \rho^{max} \geq \min(\dfrac{e-2}{v-2}, \dfrac{e}{v-1}) = \dfrac{e-2}{v-2} = \dfrac{4+s +(3+m)k'}{3+s + (2+m)k'}$. 

Используя $m=s+2$, получаем $\tilde \rho^{max} - \rho_\infty = \dfrac{1}{((2+m)(k+1)-1)(m+2)}$, следовательно
$\tilde \rho^{max} > \rho_\infty > \rho_{H_{ksm}} $.
\end{proof}

Из теоремы \ref{th:ruchinski} и леммы \ref{lem:min_ro_Hksm} следует

\begin{theorem}
 Закон нуля или единицы для языка $\LL^3$ нарушается в точках $\alpha = \frac{m+3 + (2+m)k}{m+4 + (3+m)k}$, $m \geq 3$, $k \geq 0$.
В языке $\LL^3$ есть предельные точки в любой левой полуокрестности единицы.
\end{theorem}

\section{О законе нуля или единицы при иррациональных $\alpha$ для языка $\LL^3_{\infty, \omega}$}
Первой нашей гипотезой об иррациональных $\alpha$ было то, что для них $\Gna$ подчиняется закону нуля или единицы для языка $\LL^3_{\omega, \infty}$ в некоторой левой полуокрестности единицы.

Законы нуля или единицы обычно доказывают, пользуясь критерием в терминах теории игр.

Определим \textit{игру k-Pebble на графах $G$, $H$ с $n$ раундами}.
Даны два графа, $G$ и $H$, и количество раундов $n$.
Изначально у игроков есть два одинаковых набора по $k$ различных фишек, на обоих графах фишек нет.
Игра состоит в том, что два игрока, Новатор и Консерватор, по очереди перемещают фишки по вершинам графов.%, до тех пор, пока не истечёт заранее заданное число раундов (в этом случае побеждает Консерватор) или Консерватор не сделает ход, после которого подграфы, индуцированные на вершинах, на которых стоят фишки, окажутся не изоморфными (побеждает Новатор).
%Цель Новатора --- обозначить фишками неизоморфные индуцированые подграфы, а Консерватор должен ему в этом помешать.

Раунд играется так. 
Первым ходит Новатор.
Он выбирает граф и либо выкладывает на одну из его вершин одну из имеющихся у него фишек, либо перемещает одну из фишек, лежащих на графе.
Далее Консерватор берёт копию фишки, которую перемещал Новатор, и перемещает(выкладывает) её на оставшемся графе.
Если после этого в графе $G$ вершины, на которых стоят фишки $i$ и $j$, соединены ребром тогда и только тогда, когда соединены ребром вершины графа $H$, на которых стоят копии этих фишек, то начинается следующий раунд.
Иначе побеждает Новатор.
Если Новатор не выиграл за $n$ раундов, то игра заканчивается победой Консерватора.

\begin{theorem} \cite{zhukovskii2012zero}
Случайный граф $\Gna$ подчиняется закону нуля или единицы для языка $\LL^k_{\omega, \infty}$ тогда и только тогда, когда вероятность того, что в игре k-Pebble на графах $G(n, n^{-\alpha})$, $G(m, m^{-\alpha})$  с бесконечным числом раундов у Консерватора есть выигрышная стратегия (Консерватор может играть бесконечно) стремится к 1 при $n,m \rightarrow \infty$.
\end{theorem}

Поскольку диаметр графа $\Gna,~ \alpha \in (2/3, 1)$ а.п.н. равен $d \in \left(\frac{1}{1-\alpha}, \frac{2-\alpha}{1-\alpha}\right)$ \cite{bollobas2001random}, то есть стремится к бесконечности при $\alpha \rightarrow 1 - 0$, то кажется, что у Консерватора а.п.н. должна быть выигрышная стратегия в игре 3-Pebble.
Однако мы не смогли продвинуться в этом направлении и решили попробовать доказать обратное.

Один из способов доказать отсутствие закона нуля  или единицы при некотором $\alpha$ для языка с предложениями бесконечной длины, в случае, когда свойство содержать подграф с плотностью $1/\alpha$ напрямую невыразимо в этом языке (что, разумеется, и имеет место, если $\alpha$ иррационально), состоит в следующем.
\def \seql {\{H_l\}_{l=1}^\infty}
Строится последовательность $\seql$ строго сбалансированных графов, $|V(H_{k+1})| > |V(H_{k})|$, плотности которых приближаются слева к $1/\alpha$.
% Поскольку $H_k$ обычно имеют вид своего рода цепочки, будем называть индекс $l$ \textit{длиной} графа $H_l$.
Обозначим $v_l = |V(H_l)|$.
Свойство, вероятность обладать которым не сходится ни к нулю, ни к единице (а точнее, вообще не сходится), формулируется так: ``Если $H_{l^{max}}$~--- наибольший из графов из $\seql$, имеющих копию в $\Gna$, то $v^{max} = v_{l^{max}} \in A = \bigcup_{i=1}^\infty [a_{2i-1},a_{2i}]$''. Или, говоря другими словами, число  вершин $v^{max}$ наибольшего графа из нашей последовательности, содержащегося в случайном графе $\Gna$, принадлежит объединению некоторых интервалов, подобранных таким образом, чтобы c ростом $N$ вероятность попадания $v^{max}$ в это объединение колебалась между нулём и единицей.

Действительно, поскольку все графы из $\seql$ имеют $\rho^{max} < 1/\alpha$, то для каждого $l \in \N$ вероятность того, что $H_l \subset \Gna$, стремится  к единице при $N \rightarrow \infty$, однако любой подграф случайного графа $\Gna$ имеет не более $N$ вершин.
Поэтому для больших $N$ случайный граф с высокой вероятностью содержит копии графов $H_l,~ l < l_N$, где $l_N$ растёт с ростом $N$, но с вероятностью 1 не содержит копий $H_l$ таких, что $v_l > N$.
Сходимость плотностей к $1/\alpha$ необходима для того, чтобы при добавлении в $H_l$ новых рёбер $\rho^{max}$ становилась больше $1/\alpha$, что, в свою очередь, нужно для того, чтобы сопоставить графу $H_l$ формулу $\varphi_l \in \LL^k$, которая верна (с вероятностью, стремящейся к единице) только когда $\Gna$ содержит копию $H_l',~ l' \geq l$ (для этого при доказательстве теоремы \ref{th:my limiting points} нужна была лемма \ref{lem:min_ro_Hksm}).

Теперь добавим конкретики в предыдущие рассуждения и представим схему доказательства того, что закон нуля или единицы нарушается в некоторых иррациональных точках.

Рассмотрим граф, изображённый на Рис.~\ref{fig:first block}. Обозначим его $H_1$.
\begin{figure}
  \centering
  \includegraphics[scale=0.5]{picrel/first_block.png}
  \caption{Первый блок}
  \label{fig:first block}
\end{figure}

Если присоединить к нему кусочек, как на рисунке \ref{fig:2 blocks}, то плотность и максимальная плотность графа не изменятся.

\begin{figure}
  \centering
  \includegraphics[scale=0.5]{picrel/2_blocks.png}
  \caption{Два блока}
  \label{fig:2 blocks}
\end{figure}

Можно присоединить 9 кусочков, увеличив число рёбер и 
вершин в десять раз, затем добавить (или не добавлять) одно ребро. Получившийся граф будет иметь вид, как граф на рисунке \ref{fig:additional edge}.

Таким способом можно увеличивать число рёбер и вершин в $10^k$ раз и добавлять ребро.
Последовательность плотностей графов будет стремиться к бесконечной апериодической десятичной дроби.

\begin{figure}
  \includegraphics[scale=0.5]{picrel/additional_edge.png}
  \caption{Несколько блоков и дополнительное ребро}
  \label{fig:additional edge}
\end{figure}

Граф состоит из примыкающих друг к другу секций (кусочков), поэтому для записи формулы, выражающей существование этого или более плотного графа, достаточно столько переменных, сколько нужно, чтобы выразить существование наибольшей из секций.


Пусть $1/\alpha$~--- иррациональное число вида $1.1\{z_n\}_{n=1}^\infty$, где $\{z_n\}_{n=1}^\infty$~--- произвольная апериодическая последовательность нулей и единиц. 
Например, $1/\alpha = 1.1010010001\ldots$~.
Определим для каждого $l \in \N$ граф $H_l$ следующим образом.
Для $l = 1$ граф $H_1$ с плотностью $\rho(H_1) = \rho^{max}(H_1) = 22/20 = 1.1$  уже определён.
Пусть $l > 1$.
С помощью приёма, описанного выше, увеличим число рёбер и вершин графа $H_1$ в 
%$l$ раз и дополнительно добавим $2\cdot(1/\alpha - 1.1)\cdot 10^{[\log_{10}l]}$ рёбер
$10^{l-1}$ раз и дополнительно добавим $[2\cdot(1/\alpha - 1.1)\cdot 10^{l}]$ рёбер.
Здесь необходимо выбрать такое правило добавления дополнительных рёбер, чтобы была справедлива лемма \ref{lem:balanced}.
Пока оставим эту техническую проблему в стороне и двинемся дальше, считая, что условие леммы \ref{lem:balanced} выполнено.

\begin{Lem}
\label{lem:balanced}
$\forall l \in \N $ граф $H_l$ строго сбалансирован.
\end{Lem}

Аналогично тому, как в разделе \ref{sec:triangles} для графа $H_{ksm}$ была определена формула $\varphi_{ksm}$, определим для графа $H_l$ формулу $\varphi_l$.
Мы не описываем здесь точный вид этой формулы, поскольку нам от неё важно лишь то, чтобы она была а.п.н. верна, если случайный граф содержит $H_l$, удовлетворяла лемме \ref{lem:density_irrational} и записывалась с помощью равномерно ограниченного по $l$ числа переменных. 
При этом представить себе формулу, удовлетворяющую первому и последнему требованиям, несложно, а лемма \ref{lem:density_irrational} нами не доказана, но мы считаем, что её утверждение довольно естественно.

\begin{Lem} 
\label{lem:density_irrational}
Если для графа $G$ истинна $\varphi_l$, и $G$ не содержит подграфа, изоморфного $H_l$, то $G$ содержит более плотный подграф с меньшим числом вершин.
\end{Lem}

% И тут мы сталкиваемся с проблемой~--- леммы \ref{lem:density_irrational} не достаточно для того, чтобы было истинно $\neg \left( \Gna \vDash \varphi_{N'}\right),~ N' > N$.
% Поскольку ни теорема \ref{th:ruchinski}, ни другие известные нам результаты не дают гарантий того, что $\Gna$ не содержит (с близкой к 1 вероятностью) большого (например, содержащего $N/2$ вершин) подграфа $\tilde H$ с $\rho(\tilde H) > 1/\alpha$, то мы не можем быть уверены в том, что для $\Gna$ ложна $\varphi_{N'},~ N' > N$, т.к. $\varphi_{N'}$ может быть истинна для некоторого плотного графа $\tilde H$ с количеством вершин, меньшим $N$.

% Можно попробовать решить эту проблему, добавив для каждой переменной формулы $\varphi_l$ условие, что она не соответствует никакой другой позиции графа $H_l$.
% Действительно, если $G \vDash \varphi_{N'},~ N' > N$, то в $G$ есть вершина, которой соответствуют более одной переменной* формулы  $\varphi_{N'}$. Тем не менее, вводя такие ограничения, нужно убедиться, что они не сильно уменьшают вероятность истинности формулы, в том случае, когда $H_l$ действительно содержится в $\Gna$.

% Если если эта проблема решена, то 
Следующий шаг состоит в построении множества интервалов $\{[a_{2i}, a_{2i+1}] ~ | ~ i \in \N \}$.
Будем строить его таким образом, чтобы $a_1 < a_2 < a_3 < \ldots$.

Из теоремы \ref{th:ruchinski} следует, что для любого целого числа $v$ существует число $N_{v}$ такое, что $G\left(N_{v}, N_{v}^{-\alpha}\right)$ содержит копию любого графа с $\rho^{max} < 1/\alpha$ на не более чем $v$ вершинах с вероятностью $> 0.995$.
По лемме \ref{lem:balanced} максимальная плотность всех графов из $\{H_l\}_{l=1}^\infty$ меньше $1/\alpha$.
Тогда с вероятностью $> 0.995$ при $N \in [N_{v_1}, 10N_{v_1}]$
наибольший граф из $\{H_l\}_{l=1}^\infty$, имеющий копию в $\Gna$, имеет вершин $v^{max} \in [v_1, 10N_{v_1}]$.
Положим $a_1 = v_1, ~ a_2 = 10N_{v_1}$.
Далее, при $N \in [N_{100N_{v_1}}, 10N_{100N_{v_1}})$
с вероятностью $> 0.995$ выполнено $v^{max} \in [100N_{v_1},  10N_{100N_{v_1}})$~--- этот интервал не будем включать в $A$.
Зато включим в $A$ отрезок $[a_3, a_4] =  [10N_{100N_{v_1}}, 10N_{10N_{100N_{v_1}}}] $: в него, с вероятностью $> 0.995$, попадает $v^{max}$ при $N \in [N_{10N_{100N_{v_1}}}, 10N_{10N_{100N_{v_1}}}]$.
И так продолжаем до бесконечности.

\begin{figure}
\tikz[scale=0.9, every node/.style={scale=0.9}] {
    \draw (-1.5, 0) node {$v^{max}$};
    \draw (-0.5,0) --  (14,0);
        \draw (0,0) node {\textbf[ };
            \draw (0, 0.5) node {$v_1$};
        \draw (3,0) node {\textbf] };
            \draw (3.3, 0.5) node {$10\bfN_{v_1}$};
        \draw (5,0) node {\textbf[ };
            \draw (5, 0.5) node {$100\bfN_{v_1}$};
        \draw (7.95,0) node {\textbf) };
        \draw (8,0) node {\textbf[ };
            \draw (8.7, 0.5) node {$10\bfN_{100\bfN_{v_1}}$};
        \draw (12.5,0) node {\textbf] };
            \draw (13.5, 0.5) node{$10\bfN_{10\bfN_{100\bfN_{v_1}}}$};
    \draw (-1.5, -2) node {$N$};
    \draw (-0.5,-2) --  (14,-2);
        \draw (2,-2) node {\textbf[ };
            \draw (2, -2.5) node {$\bfN_{v_1}$};
        \draw (3,-2) node {\textbf] };
            \draw (3.3, -2.5) node {$10 \bfN_{v_1}$};
        \draw (7,-2) node {\textbf[ };
            \draw (6.7, -2.5) node {$\bfN_{100\bfN_{v_1}}$};
        \draw (8,-2) node {\textbf) };
            \draw (8.7, -2.5) node {$10\bfN_{{100\bfN_{v_1}}}$};
        \draw (11,-2) node {\textbf[ };
            \draw (11, -2.5) node {$\bfN_{10\bfN_{100\bfN_{v_1}}}$};
        \draw (12.5,-2) node {\textbf] };
            \draw (13.5, -2.5) node {$10\bfN_{10\bfN_{100\bfN_{v_1}}}$};
            
    \draw[->, thick] (2.5, -1.8) -- node[midway, left] {$P>0.99$} (1.5, -0.2);
    \draw[->, thick] (7.5, -1.8) -- node[midway, left] {$P>0.99$} (6.5, -0.2);
    \draw[->, thick] (11.75, -1.8) -- node[midway, left] {$P>0.99$} (10.5, -0.2);
    \draw[decorate, decoration=snake, blue, thick] (0, 0) -- (3,0);
    \draw[decorate, decoration=snake, gray, thick] (5, 0) -- (8,0);
    \draw[decorate, decoration=snake, blue, thick] (8, 0) -- (12.5,0);
}
\caption{На верхней оси изображены интервалы $[a_i, a_{i+1}] \subset A$ (выделены голубым) и интервалы, намеренно исключенные из $A$ (выделены серым). На нижней оси изображены интервалы числа вершин случайного графа $\Gna$, на которых $v^{max}$ с высокой вероятностью принадлежит соответствующим интервалам на верхней оси }
\end{figure}


Пусть теперь $N'_v$~--- такое число, что при $N = N'_v$ случайный граф $\Gna$ с вероятностью $>0.995$ не содержит графов на не более чем $v$ вершинах с $\rho^{max} > 1/\alpha$.
Если положить $\mathbf{N}_v = \max\{N_v, N'_v\}$, и заменить в рассуждении выше $N_v$ на $\mathbf{N}_v$, то с помощью полученных таким образом интервалов можно построить формулу, вероятность истинности которой не будет иметь предела в $[0,1]$.
Действительно, при $N > \mathbf{N}_v$ по лемме \ref{lem:density_irrational} с вероятностью, превышающей $0.99$ ($\P(B \cap C) \geq 1 - (\P(\overline B) + \P(\overline C))$), свойство содержать $H_l:~ |V(H_l)| \leq v$ равносильно истинности $\varphi_l$.
В последовательности $\{H\}_{l=1}^\infty$ между индексами $l$ и числом вершин $v_l$ существует однозначное соответствие и, по построению, $v_{l+1} = 10 v_l$.
Интервалы, входящие в $A$, а также интервалы, которые мы намеренно исключили из $A$, строились таким образом, чтобы их нижние границы отличались от верхних в $10$ раз, поэтому для каждого из этих интервалов существует индекс $l$ такой, что $v_l$ попадает в интервал.
Поэтому объединению интервалов числа вершин $A$ соответствует множество индексов $L$ такое, что вероятность истинности формулы \[
\bigvee_{l \in L} \varphi_l \wedge \neg \varphi_{l+1} 
\]
не имеет предела при $N \rightarrow \infty$.

В этом рассуждении остаются пробелы в виде доказательств лемм \ref{lem:balanced}, \ref{lem:density_irrational}.
Если бы они были устранены, мы бы доказали, что в любой левой полуокрестности единицы существует иррациональное $\alpha$, для которого нарушается закон нуля или единицы для языка $\LL^k_{\infty, \omega}$ при некотором $k$ порядка $10$.
Далее можно было бы оптимизировать конструкцию, уменьшив число переменных в формуле $\varphi_l$, и доказать это утверждение для $k$, близкого или равного трём.

\chapter{Заключение}
//TODO

\backmatter
\bibliographystyle{unsrt}
\bibliography{bib}

\end{document}