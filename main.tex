\documentclass{mipt-thesis-bs}
    % Следующие две строки нужны только для biblatex. Для inline-библиографии их следует убрать.
    %\usepackage{mipt-thesis-biblatex}
    %\addbibresource{bib.bib}

    \title{Спектры предложений первого порядка с ограниченным количеством переменных}
    \author{Ярмошик Д.\,В.}
    \supervisor{Жуковский М.\,Е.}
    %\referee{Петров Д.\,Е.}       % требуется только для mipt-thesis-ms
    \groupnum{573}
    \faculty{Факультет управления и прикладной математики}
    \department{Кафедра «Интеллектуальные системы», специализация «Проектирование и организация систем»}

    \begin{document}

    \frontmatter
    \titlecontents

    \mainmatter


    \chapter{Введение}
    Случайные графы -- один из центральных разделов дискретной математики, расположенный на стыке теории вероятностей, комбинаторики и теории графов.
    Основы теории случайных графов были заложены в 50-х -- 60-х годах прошлого века венгерскими математиками П.~Эрдёшем и А.~Реньи.
    Существует множество моделей случайных графов, разработанных с учётом адекватности их применения в прикладных областях.
    В настоящей работе мы будем иметь дело с \textit{моделью Эрдёша-Реньи}, также называемой \textit{классической моделью} случайного графа.
    % Определение G(N,p)
    
    Многие вопросы этой теории связаны с асимптотическими свойствами случайного графа, то есть с тем, как он ведёт себя при устремлении количества вершин $N$ к бесконечности.
    Результаты исследования ряда этих вопросов в значительной степени обобщаются так называемыми \textit{законами нуля или единицы} -- теоремами, утверждающими для определённого случайного графа про любое свойство из некоторого класса свойств можно заранее сказать, что при $N \rightarrow \infty$ граф либо будет обладать этим свойством с вероятностью, стремящейся к единице, либо с вероятностью, стремящейся к единице, граф этим свойством обладать не будет.
    
    Введём обозначение $\mathcal{L}$ для языка первого порядка с сигнатурой, в которую входят только двуместные предикаты равенства ($=$) и смежности ($\sim$).
    За точными определениями можно обратиться, например, к
    \cite{shen}, а здесь мы ограничимся лишь напоминанием о том, что формулы языка первого порядка -- это предложения, составленные из символов, обозначающих переменные: 
    $x,y,z,x_1,\ldots$, 
    логических связок 
    $\wedge, ~\neg, ~\vee$, 
    кванторов 
    $\exists, ~\forall$ и 
    предикатных символов (в нашем случае
    $\sim, ~ =$).
    Например, формула, утверждающая, что диаметр графа не превосходит 2, выглядит так:
    \[
    \forall x \forall y ~ x=y \vee x\sim y \vee \exists z ~ x \sim z \wedge z \sim y
    \]
    
    \
    
    



    \backmatter
    \bibliographystyle{unsrt}
    \bibliography{bib}
    %\printbib
    % Следующие строки необходимо раскомментировать, а предыдущую закомментировать, если используется inline-библиография.
    %\begin{thebibliography}{99}
    %    \bibitem{langmuir26}
    %        H. Mott-Smith, I. Langmuir. ``The theory of collectors in gaseous discharges''. \emph{Phys. Rev.} \textbf{28} (1926)
    %\end{thebibliography}

    \chapter{Благодарности}

    Благодарности идут тут.

    \end{document}