Нас интересуют выразительные способности языков $\LL^k$ и $\LL^k_{\infty, \omega}$ и, в частности, наиболее простых среди них~--- языков  $\LL^3$ и $\LL^3_{\infty, \omega}$.
Из результата МакАртур \cite{mcarthur1997asymptotic} следует, что для этих языков выполнен закон нуля или единицы при $\alpha < \frac{1}{2}$.
Кроме того, известно (М. Жуковский, А. Раджафимахатратра, 2019 \cite{razafimahatratra2019zero}), что при $\alpha = \frac{1}{k-1}$ закон нуля или единицы для языка $\LL^k$ всё ещё справедлив, но в любой правой полуокрестности $\frac{1}{k-1}$ существует $\alpha$, для которого закон нуля или единицы нарушается.
Что можно сказать о сохранении закона нуля или единицы для языков с тремя переменными при других значениях параметра $\alpha$, например в окрестности $\alpha=1$~--- точки, при переходе через которую граф теряет связность?
Существуют утверждения, не выразимые на языке $\LL$, но выразимые в $\LL^3_{\infty, \omega}$ и наоборот.
Для языка $\LL$ закон нуля или единицы выполнен при всех иррациональных $\alpha$, верно ли это для $\LL^3_{\infty, \omega}$?
Различается ли структура множеств параметров, на которых нарушается закон нуля или единицы, у языков с ограниченным количеством переменных и языков с ограниченной кванторной глубиной? 