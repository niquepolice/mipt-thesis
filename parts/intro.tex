    Случайные графы -- один из центральных разделов дискретной математики, расположенный на стыке теории вероятностей, комбинаторики и теории графов.
    Основы теории случайных графов были заложены в 50-х -- 60-х годах прошлого века венгерскими математиками П.~Эрдёшем и А.~Реньи.
    Существует множество моделей случайных графов, разработанных с учётом адекватности их применения в прикладных областях: моделирования социальных, биологических и инфраструктурных сетей. 
    В настоящей работе мы будем иметь дело с \textit{моделью Эрдёша-Реньи}, также называемой \textit{классической моделью} случайного графа.
    % Определение G(N,p)
    
    Многие вопросы этой теории связаны с асимптотическими свойствами случайного графа, то есть с тем, как он ведёт себя при устремлении количества вершин $N$ к бесконечности.
    Удивительным образом оказывается, что для целого класса таких вопросов можно, совершенно не вдаваясь в сущность конкретного вопроса, дать на него едва ли не исчерпывающий ответ.
    Точнее говоря, если про случайный граф спрашивается ,,Как, при увеличении числа вершин до бесконечности, будет вести себя вероятность того, что граф обладает свойством $A$?`` то во многих случаях можно заранее сказать: ,,Будет стремиться к 0 или к 1``.
    В этом состоит суть теорем, называемых \textit{законами нуля или единицы}, уточнению условий применимости которых посвящена данная работа.
    
    Чтобы перейти к изложению известных результатов о законах нуля или единицы, дадим определения основным понятиям и сформулируем теоремы, касающиеся важных для нас свойств случайного графа $G(N,p)$ (см. \cite{survey2015})
    
    Пусть $N \in \N$, $0 \leq p \leq 1$. Рассмотрим множество $\Omega_N = \{G = (V_N , E)\}$, $|\Omega_N| = 2^{C_N^2}$ всех неориентированных графов без петель и кратных ребер с множеством вершин $V_N = \{1, \ldots, N \}$.
    
    \Def \textit{Случайный граф $G(N, p) \in \Omega_N$  в модели Эрдёша-Реньи} -- это элементарный исход в вероятностном пространстве $(\Omega_N, P_{N.p}, 2^{\Omega_N})$, где  распределение $P_{N.p}$ определяется формулой $P_{N.p}(G) = p^{|E|}(1-p)^{C_N^2-|E|}$
    
    \Def Для произвольного языка $\F$ , случайный граф $G(N, p)$ \textit{подчиняется закону нуля или единицы для языка $\F$},
    если для любой формулы $\varphi$ из языка 
    $\F$  выполнено
    $\lim_{N \rightarrow \infty} P(G(N, p) \vDash \varphi) \in \{0, 1\}$.
    
    Введём обозначение $\LL$ для языка первого порядка с сигнатурой, в которую входят только двуместные предикаты равенства ($=$) и смежности ($\sim$).
    За точными определениями можно обратиться, например, к
    \cite{shen}, а здесь мы ограничимся лишь напоминанием о том, что формулы языка первого порядка -- это предложения, составленные из символов, обозначающих переменные: 
    $x,y,z,x_1,\ldots$, 
    логических связок 
    $\wedge, ~\neg, ~\vee$, 
    кванторов 
    $\exists, ~\forall$ и 
    предикатных символов (в нашем случае
    $\sim, ~ =$).
    Например, формула, утверждающая, что диаметр графа не превосходит 2, выглядит так:
    \[
    \forall x \forall y ~ x=y \vee x\sim y \vee \exists z ~ x \sim z \wedge z \sim y
    \]
    
    \Def \textit{Плотность} графа $G$ 
   \[ \rho_G = \frac {|E(G)|}{|V(G)|} \]
   
   Здесь $E(G)$ -- множество вершин графа $G$, $V(E)$ -- множество рёбер
   
   \Def \textit{Максимальная плотность} графа $G$
     \[ \rho^{max}_G = \max_{H \subseteq G} \frac {|E(H)|}{|V(H)|} \]
    
    \Def Граф $G$ называется \textit{сбалансированным}, если 
    \[ \rho^{max}_G = \rho_G \]
    
    \Def Граф $G$ называется \textit{строго сбалансированным}, если 
    \[\rho^{max}_G = \rho_G > \max_{H \subsetneq G} \frac {|E(H)|}{|V(H)|} \]
     
В 1960 г. П. Эрдёш и А. Реньи
доказали теорему о об условиях справедливости утверждения “$G(N, p)$ содержит копию данного сбалансированного графа” \cite{erdHos1976evolution}. 
Позже этот результат получил следующее обобщение
    
\begin{theorem} (А. Ручински, А. Винс, 1985, \cite{rucinski1985balanced}) 
\label{th:ruchinski}
Вероятность того, что $G(N, p)$ имеет подграф $H$ стремится к 1 при $N \rightarrow \infty$, если $N^{-1/\rho^{max}(H)} = o(p(N))$ и стремится к  0, если $p(N) = o\left(N^{-1/\rho^{max}(H)}\right)$

\end{theorem}

Для нас очень важно то, что при $p = N^{-1/\rho^{max}(H)}$ вероятность содержать подграф $H$ не стремится ни к нулю, ни к единице.
Это утверждение, содержащееся в следующей теореме, даёт удобный способ доказывать отсутствие законов нуля или единицы для $G(N, N^{-\alpha}), ~\alpha \in \Q$

\begin{theorem} (Б. Боллобаш, 1981, \cite{bollobas1981threshold}). Пусть $H$~-- строго сбалансированный граф, $a$~-- количество автоморфизмов графа $H$, $p = N^{-1/ \rho^{max}_H}$
Тогда
\[N_H \xrightarrow[N\rightarrow \infty]{d} Poiss(1/a) \]
Здесь $N_H$ -- количество копий графа $H$ в $G(N, p)$, $Poiss(1/a)$ -- пуассоновская случайная величина со средним $1/a$.
\end{theorem}
Из последней теоремы и доказанного в \cite{rucinski1986strongly} утверждения о том, что для любого $\alpha \in (0,1] \cap \Q$ существует граф $H$ с плотностью $\rho_H = 1/\alpha$, следует, что $G(N,N^{-\alpha})$ не подчиняется закону нуля или единицы для языка $\LL$ при всех рациональных $\alpha$ из отрезка $[0,1]$.
Это верно и для $\alpha = (k+1)/k~, k > 0$, т.к. для каждого из таких $\alpha$ существует дерево с соответствующей плотностью.
При $\alpha > 2$ граф $G(N, N^{-\alpha})$ почти наверное пуст, и, следовательно, подчиняется закону нуля или единицы.
Оказывается, что при всех прочих $\alpha$ случайный граф $G(N, N^{-\alpha})$ также подчиняется закону нуля или единицы.
Итак, имеет место следующая теорема
\begin{theorem} (Дж. Спенсер, С. Шела, 1988, \cite{shelah1988zero})
Случайный граф $G(N, N^{-\alpha})$ подчиняется закону нуля или единицы для языка $\LL$ при всех $\alpha$, кроме $\alpha \in \left([0,1] \cap \Q \right) \cup \{(k+1)/k ~|~ k > 1\}$.
\end{theorem}

Ниже мы сформулируем несколько аналогичных теорем для других языков.

\Def Язык $\LL^k$ -- подмножество $\LL$, содержащее предложения, в которые входят не более $k$ переменных.

На языке $\LL^3$ для любого натурального $d$ можно выразить, например, свойство ``иметь диаметр не более $d$'': 
\[
\forall x \forall y ~ x = y \vee x \sim y \vee \left( \exists z ~ x\sim z \wedge z \sim y \right)
\vee  \left( \exists z ~ x\sim z \wedge \exists x ~ z \sim x \wedge x \sim y \right) \vee \ldots
\]

\Def Язык $\LL^k_{\infty, \omega}$ включает в себя предложения конечной или счётной длины, в которые входят не более $k$ переменных.

На языке $\LL^3_{\infty, \omega}$ можно выразить, например, свойство ``быть связным''.
Это свойство невыразимо в $\LL$.

\Def \[\LL^\omega_{\infty, \omega} = \bigcup_k \LL^k_{\infty, \omega} \]

\begin{theorem}
Случайный граф $G(N, N^{-\alpha})$ подчиняется закону нуля или единицы для языка $\LL^\omega_{\infty, \omega}$ при $\alpha \in (1,2] \setminus \{(k+1)/k ~|~ k > 1\}$ (Дж. Линч, 1993  \cite{lynch1993infinitary}) и $\alpha > 2$.
Закон нуля или единицы не выполнен при $\alpha \in [0,1]$ (С. Шела, 2017 \cite{shelah2017failure})  и $\alpha \in \{(k+1)/k ~|~ k > 1\}$.
\end{theorem}

\begin{theorem} (М. Mакартур, 1997 \cite{mcarthur1997asymptotic})
Для любых $\alpha < \dfrac{1}{k-1}$, $G(N, N^{-\alpha})$ подчиняется закону нуля или единицы для языка $\LL^k_{\infty, \omega}$.
\end{theorem}

\Def \textit{Кванторная глубина} формулы -- наибольшее число вложенных кванторов в формуле.

\Def Язык $\LL_k$ - подмножество $\LL$, включающее формулы с кванторной глубиной не более $k$.

\begin{theorem}(М. Е. Жуковский, 2012, [60]). Пусть $p=N^{-\alpha}$, $0 < \alpha < 1/(k - 2)$. 
Тогда случайный граф $G(N, p)$ подчиняется закону нуля или единицы для языка $\LL_k$.
\end{theorem}