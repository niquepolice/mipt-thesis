
\Def \textit{Спектром} формулы $\varphi$ называется множество $\alpha$ таких, что вероятность $\P\left( G\left(N, N^{-\alpha}\right) \vDash \varphi  \right)$ не стремится ни к 0, ни к 1 при $N \rightarrow \infty$.
Здесь использовано обозначение $G \vDash \varphi$, означающее, что для графа $G$ истинна $\varphi$.

\begin{theorem}
В языке $\LL^3_{\infty, \omega}$ существует формула с бесконечным спектром с предельной точкой 1 (справа).
\end{theorem}
\begin{proof}
Обозначим $\varphi_m$ замкнутую формулу, которая истинна тогда и только тогда, когда в графе существует цикл длины не более $m$  или простой путь длины $m$,  $\varphi_m \in \mathcal{L}^3$. Например,
\[
\varphi_3 = \exists x ~\exists y ~
 x \sim y  \wedge \left( \exists z ~ y \sim z \wedge  z \neq x 
\wedge \left(\exists x ~ x \sim z \wedge x \neq y
\right) \right) 
\]
%\textstyle\bigwedge\limits
Определим $\varphi \in \mathcal{L}^3_{\infty, \omega}$ :
\[
\varphi = \bigvee_{k = 2}^{+\infty}\left(
\wedge_{i=3}^{2k-1} \varphi_{i}  \wedge \neg \varphi_{2k} \right)
\]

Пусть $\alpha = 1 + \frac{1}{2k}$.
Напомним, что при таком $\alpha$, по теореме \ref{th:ruchinski}, вероятность обнаружить в случайном графе $G(N, N^{-\alpha})$ подграф $H$ с плотностью $\rho_H > \frac{2k}{2k+1}$ стремится к нулю, с плотностью $\rho_H < \frac{2k}{2k+1}$ стремится к 1, а с плотностью $\rho_H = \frac{2k}{2k+1}$ -- ни к нулю ни к единице.
Поскольку $\frac{2k}{2k+1}$ -- это плотность простого пути длины $2k$, то асимптотически почти наверное случайный граф $G(N, N^{-\alpha})$ содержит простой путь длины $2k-1$ , поэтому в формуле $\varphi$ дизъюнкция конъюнктов до $k-1$ -го включительно ложна с вероятностью, стремящейся к 1.
Также, с вероятностью, стремящейся к 1, $G(N, p)$ не содержит циклов длины не более $2k+1$ и простых путей длины $2k+1$.
Поэтому дизъюнкция конъюнктов от $k+1$-го до бесконечности ложна с вероятностью, стремящейся к 1.

Количество простых путей длины $2k$ в $G(N, N^{-\frac{2k+1}{2k}})$ имеет асимптотическое распределение $Poiss(1/2)$, поэтому
$$
\P(G(n, p) \models \varphi) = 
\P(G(n, p) \models \neg \varphi_{2k}) \rightarrow e^{-1/2}
$$

Аналогично, при $\alpha = 1 + \frac{1}{2k+1}$ 
$$\P(G(n, p) \models \varphi) = 
\P\left(G\left(n, p \right) \models  \varphi_{2k+1}\right) \rightarrow 1 - e^{-1/2}$$

Таким образом, спектр формулы $\varphi$ содержит множество $\{1 + 1/k ~|~ k > 5 \}$, что и требовалось.
\end{proof}